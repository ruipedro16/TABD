\subsection{Number of Athletes by Age}\label{subsec:number-of-athletes-by-age}

To determine to determine the average age of the athletes, we can run the following SQL query:

\begin{minted}{sql}
SELECT AVG(age(birthdate))
FROM annp_final.athlete;
\end{minted}

From this, we can see that the average age of the athletes is 46 years, 6 months and 31 days.

We can also determine who's the youngest athlete by running the following SQL query:

\begin{minted}{sql}
SELECT *
FROM annp_final.athlete
ORDER BY age(birthdate) ASC
LIMIT 1;
\end{minted}

\begin{itemize}
    \item \textbf{Name:} Ana Mónica Eloi
    \item \textbf{Gender:} F
    \item \textbf{Birthdate:} 29/12/1996
    \item \textbf{Age:} 25 years
\end{itemize}

On the other hand, we can learn information about the oldest athlete by running the following SQL query:

\begin{minted}{sql}
SELECT *
FROM annp_final.athlete
ORDER BY age(birthdate) DESC
LIMIT 1;
\end{minted}

\begin{itemize}
    \item \textbf{Name:} Virgílio Zacarias Costa
    \item \textbf{Gender:} M
    \item \textbf{Birthdate:} 21/07/1931
    \item \textbf{Age:} 90 years
\end{itemize}

Finally, to determine the number of athletes by age, we can run the following SQL query using the PostgreSQL's built-in
\texttt{age} function:

\begin{minted}{sql}
SELECT COUNT(*), EXTRACT(YEAR FROM age(birthdate)) AS age
FROM annp_final.athlete
GROUP BY age
ORDER BY age ASC;
\end{minted}

We can then plot the result, as illustrated in \cref{fig:athletesbynation}.

\begin{figure}[H]
    \centering
    \includegraphics[width=\textwidth]{img/athletesbyage}
    \caption{Number of athletes by age}
    \label{fig:athletesbyage}
\end{figure}

\subsection{Number of Athletes by Nation}\label{subsec:number-of-athletes-by-nation}

To determine the number of athletes by nation, we can run the following SQL query:

\begin{minted}{sql}
SELECT COUNT(*) nationCount, nation
FROM annp_final.athlete
GROUP BY nation
ORDER BY nationCount ASC;
\end{minted}

We can then plot the result, as illustrated in \cref{fig:athletesbynation}.

\begin{figure}[H]
    \centering
    \includegraphics[width=.8\textwidth]{img/athletesbynation}
    \caption{Number of athletes by nation}
    \label{fig:athletesbynation}
\end{figure}

To have another perspective, we can also plot in a pie chart, as illustrated in \cref{fig:athletesbynation-pie}.

\begin{figure}[H]
    \centering
    \includegraphics[width=.8\textwidth]{img/athletesbynation-piechart}
    \caption{Number of athletes by nation}
    \label{fig:athletesbynation-pie}
\end{figure}

\subsection{Number of Athletes by Gender}\label{subsec:number-of-athletes-by-gender}

To determine the number of athletes by gender, we can run the following SQL query:

\begin{minted}{sql}
SELECT COUNT(*), gender
FROM annp_final.athlete
GROUP BY gender;
\end{minted}

We can then plot the result, as illustrated in \cref{fig:athletesbygender}.

\begin{figure}[H]
    \centering
    \includegraphics[width=.35\textwidth]{img/athletesbygender}
    \caption{Number of athletes by gender}
    \label{fig:athletesbygender}
\end{figure}

We can also plot this in a pie chart, as illustrated in \cref{fig:athletesbygender-pie}.

\begin{figure}[H]
    \centering
    \includegraphics[width=.45\textwidth]{img/athletesbygender-pie}
    \caption{Percentage of athletes by gender}
    \label{fig:athletesbygender-pie}
\end{figure}

\subsection{Number of Events by Gender}\label{subsec:number-of-events-by-gender}

To determine the number of events by gender, we can run the following SQL query:

\begin{minted}{sql}
SELECT COUNT(*), gender
FROM annp_final.event
GROUP BY gender;
\end{minted}

We can then plot the result, as illustrated in \cref{fig:eventsbygender}.

\begin{figure}[H]
    \centering
    \includegraphics[width=.35\textwidth]{img/eventsbygender}
    \caption{Number of events by gender}
    \label{fig:eventsbygender}
\end{figure}

Here, the value \texttt{X} refers to events that allow athletes from both genders to participate.
We can also plot this in a pie chart, as illustrated in \cref{fig:eventsbygender-pie}.

\begin{figure}[H]
    \centering
    \includegraphics[width=.45\textwidth]{img/eventsbygender-pie}
    \caption{Percentage of events by gender}
    \label{fig:eventsbygender-pie}
\end{figure}

\subsection{Number of Clubs by Nation}\label{subsec:number-of-clubs-by-nation}

We can determine the number of clubs by each nation by running the following SQL query:

\begin{minted}{sql}
SELECT nation, COUNT(*) AS nationCount
FROM annp_final.club
GROUP BY nation
ORDER BY nationCount ASC;
\end{minted}

\textcolor{red}{COMPLETAR COM TEXTO}.

\begin{figure}[H]
    \centering
    \includegraphics[width=.35\textwidth]{img/clubsbynation}
    \caption{Number of clubs by nation}
    \label{fig:clubs-by-nation}
\end{figure}

\textcolor{red}{COMPLETAR COM TEXTO}.

\begin{figure}[H]
    \centering
    \includegraphics[width=.45\textwidth]{img/clubsbynation-pie}
    \caption{Percentage of clubs by nation}
    \label{fig:clubs-by-nation-pie}
\end{figure}

\subsection{Number of Clubs by Region}\label{subsec:number-of-clubs-by-region}

\textcolor{red}{COMPLETAR COM TEXTO}.

\begin{minted}{sql}
SELECT region, COUNT(*) AS regionCount
FROM annp_final.club
WHERE region SIMILAR TO '[A-Z]+'
GROUP BY region
ORDER BY regionCount ASC;
\end{minted}

\textcolor{red}{COMPLETAR COM TEXTO}.

\begin{figure}[H]
    \centering
    \includegraphics[width=.85\textwidth]{img/clubsbyregion}
    \caption{Number of clubs by region}
    \label{fig:clubs-by-region}
\end{figure}

\subsection{Swim Styles}

\textcolor{red}{COMPLETAR COM TEXTO (style com mais distance) É FRESSTYLE}.

\begin{minted}{sql}
SELECT *
FROM annp_final.swimstyle
ORDER BY distance DESC
LIMIT 1;
\end{minted}

\textcolor{red}{COMPLETAR COM TEXTO (style com menos distance) FLY}.

\begin{minted}{sql}
SELECT *
FROM annp_final.swimstyle
ORDER BY distance ASC
LIMIT 1;
\end{minted}

\subsection{Results}

\subsubsection{Average Swim Time}

\textcolor{red}{COMPLETAR COM TEXTO (resultado = 00:02:23.769068)}.

\begin{minted}{sql}
SELECT AVG(swimtime)
FROM annp_final.result;
\end{minted}

\subsubsection{Average Number of Points}

\textcolor{red}{COMPLETAR COM TEXTO (resultado = 340)}.
\begin{minted}{sql}
SELECT AVG(points)::numeric(10, 1)
FROM annp_final.result;
\end{minted}

\subsection{Club facts statistics}

\subsubsection{Overall Statistics}

Given the Club facts table, we can plot some of their statistics. Below, we present a query that fetches these statistics for all possible combinations
between \texttt{meetid} and \texttt{clubid}. Note that instead we used the \texttt{code} column from \texttt{club} table to easily read each team for 
a given tournament. 

\begin{minted}{sql}
SELECT 
CASE GROUPING(cd.meetid)
    WHEN 1 THEN 'all_meets'
    ELSE cd.meetid
END AS "Tournament",
CASE GROUPING(c.code)
    WHEN 1 THEN 'all_clubs'
    ELSE c.code
END AS "Team",
   ROUND(AVG(average_age), 0) AS "Average Age",
   ROUND(AVG(average_swimtime), 2) AS "Average Swimtime",
   ROUND(SUM(total_points)) AS "Total Points",
   ROUND(SUM(number_of_players)) AS "Total Players"
FROM (
    SELECT CAST(meetid AS VARCHAR(255)),
       clubid,
       average_age,
       total_points,
       average_swimtime,
       number_of_players
    FROM annp_final.club_defacto) cd
JOIN annp_final.club c ON c.clubid = cd.clubid
GROUP BY CUBE (cd.meetid, c.code);
\end{minted}


\begin{figure}[H]
    \centering
    \includegraphics[width=\textwidth]{img/club_fact1.pdf}
    \caption{Statistics from fact Club table.}
    \label{fig:clubs_fact}
\end{figure}

\subsubsection{Statistics given a swimstyle}

Next, we also show the overall statistics given a swim style. For that, we use the query

\begin{minted}{sql}
SELECT 
    CASE GROUPING(c.code)
        WHEN 1 THEN 'all_clubs'
        ELSE c.code
    END AS "Team",
    CASE GROUPING(cd.swimstyleid)
        WHEN 1 THEN 'all_styles'
        ELSE cd.swimstyleid
    END AS "SwimStyle",
    ROUND(AVG(average_swimtime), 2) AS "Average Swimtime",
    ROUND(SUM(total_points)) AS "Total Points",
    ROUND(SUM(number_of_players)) AS "Total Players"
FROM annp_final.club_defacto cd
JOIN annp_final.club c on c.clubid = cd.clubid
GROUP BY CUBE (c.code, cd.swimstyleid)
ORDER BY "Total Points" DESC
\end{minted}

To filter the amount of information this table has, we filter, for a given swim style, the top5 teams that had the higher total of points. This is done for all the
tournaments. This pre-processing step was done in python and the result is depicted in Figure \ref{fig:clubs_fact2}. Note that not all the bar plots have 5 teams. This
is due to the lack of data presented from both tournaments.

\begin{figure}[H]
    \centering
    \includegraphics[width=\textwidth]{img/stats_clubs_swim.pdf}
    \caption{Top 5 teams per swim style for a given statistic.}
    \label{fig:clubs_fact2}
\end{figure}

\subsection{Athlete facts statistics}

\begin{minted}{sql}
SELECT
    CASE GROUPING(a.firstname)
        WHEN 1 THEN 'all_players'
        ELSE a.firstname
    END AS "Athletes",
    CASE GROUPING(af.meetid)
        WHEN 1 THEN 'all_meets'
        ELSE af.meetid
    END AS "Tournament",
    ROUND(AVG(average_points), 2) AS "Average Points",
    ROUND(AVG(average_distance), 2) AS "Average Distance",
    ROUND(AVG(average_swimtime), 2) AS "Average Swimtime"
FROM (
    SELECT
        athleteid,
        CAST(meetid as VARCHAR(255)),
        average_points,
        average_distance,
        average_swimtime
    FROM annp_final.athlete_defacto) af
JOIN annp_final.athlete a ON a.athleteid = af.athleteid
GROUP BY CUBE (a.firstname, af.meetid)
ORDER BY "Average Points" DESC
LIMIT 50;
\end{minted}


\begin{figure}[H]
    \centering
    \includegraphics[width=\textwidth]{img/athletefact.pdf}
    \caption{Statistics from fact Club table.}
    \label{fig:athlete_fact}
\end{figure}